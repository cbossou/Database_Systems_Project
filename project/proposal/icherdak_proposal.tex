\documentclass{article}

\usepackage[utf8]{inputenc}
\usepackage{lipsum}                     % Dummytext
\usepackage{hyperref}
\usepackage{xargs}                      % Use more than one optional parameter in a new commands
\usepackage[pdftex,dvipsnames]{xcolor}  % Coloured text etc.
\usepackage{graphicx}
\usepackage{verbatim}
\usepackage{float}
\usepackage{tikz-qtree}
\usepackage{tikz}
\usepackage[linguistics]{forest}

\usepackage{amssymb}
\usepackage{amsmath}
\newcommand*{\QEDA}{\hfill\ensuremath{\blacksquare}}% filled box
\newcommand*{\QEDB}{\hfill\ensuremath{\square}}% unfilled box

% dem nice tables
\usepackage[hmargin=2cm,top=4cm,headheight=65pt,footskip=65pt]{geometry}
\usepackage{fmtcount} % for \ordinalnum
\usepackage{array,multirow}
\usepackage{tabularx}
\usepackage{lastpage}


% add a special collumn type
\newcolumntype{C}[1]{>{\centering\arraybackslash}m{#1}}


%header/footer stuff
\usepackage{fancyhdr}
\pagestyle{fancy}

%note that if you do not do these blank ones, the package defaults to something
%you may not want in your header or footer
\lhead{Project Proposal}
\chead{CMPS 278}
\rhead{\today}
\lfoot{Isaak Cherdak}
\cfoot{}
\rfoot{\thepage}

\renewcommand{\headrulewidth}{0pt}
\renewcommand{\footrulewidth}{0pt}

\hypersetup{
    colorlinks=true,
    linkcolor=blue,
    filecolor=magenta,
    urlcolor=cyan,
}

\usepackage[english]{babel}
\emergencystretch=1pt
\usepackage[justification=centering]{caption}
\graphicspath{{Pictures/} }

\usepackage[colorinlistoftodos,prependcaption,textsize=tiny]{todonotes}
\newcommandx{\unsure}[2][1=]{\todo[linecolor=red,backgroundcolor=red!25,bordercolor=red,#1]{#2}}
\newcommandx{\change}[2][1=]{\todo[linecolor=blue,backgroundcolor=blue!25,bordercolor=blue,#1]{#2}}
\newcommandx{\info}[2][1=]{\todo[linecolor=OliveGreen,backgroundcolor=OliveGreen!25,bordercolor=OliveGreen,#1]{#2}}
\newcommandx{\improvement}[2][1=]{\todo[linecolor=Plum,backgroundcolor=Plum!25,bordercolor=Plum,#1]{#2}}
\newcommandx{\thiswillnotshow}[2][1=]{\todo[disable,#1]{#2}}

\usepackage{setspace}
\doublespacing

\title{CMPS 278 Project Proposal}
\author{Isaak Cherdak}
%\date{} %blank date

\begin{document}

\maketitle

\pagebreak

\section{Overview}
\label{sec:overview}

I am interested in the prospect of working on characterization of use cases and
potential optimizations of RocksDB. I plan to try to run varying
put/get/update/delete operations on a local RocksDB instance.

\subsection{Background Experience}
\label{subsec:background}

I have experience setting up Benchmarks in C/C++ for various data structures.
Specifically, I have been able to send multiple experiment trials to a file and
have used pilot, a tool made by the SSRC (my research lab), to generate
analytical information about the results (such as standard deviation, average,
etc). I have also been able to graph these results using matplotlib. I plan to
apply this experience to design a similar process of running experiments on
RocksDB.

I previously did experiments on four different data structures: Singly Linked
List, Vector, B-Tree, and LSM-Tree. The first three were written by myself in a
consistent manner and the LSM-Tree was adapted based off one created by a
Harvard PhD student. The benchmark was at best able to run 40000 inserts and
40000 get calls with 100 trials in approximately 30 seconds. The order of
inserts and get calls wasn't the same but the way they were run differed
depending on the data structure. The hardware setup was as described in
\ref{subsubsec:test_hw} as well as my laptop which had comparable results. Test
results included aggregate runtime of all inserts, aggregate runtime of all
calls to get, and total in-memory size at the end of all insertions. It should
be noted that I was able to run these experiments without having any issues with
memory / storage usage (runtime was main bottleneck) when using an 800 MB NVRAM
pool. Valgrind reported approximately 25 GB of dynamic memory allocating/freeing
over the course of all trials/experiments in the case of 40000 inserts/100
trials.

\subsection{Benchmark}

A benchmark will be designed in C++. This benchmark will run a variety of
different tests with varying ratios of operations (put/get/update/delete). A
single experiment is defined as a single group of operations that is run with at
least 100 trials. The purpose of doing multiple trials is to ensure accuracy of
results.

\subsubsection{Characterization of tests}
\label{subsubsec:characterization_tests}

I will be running various ratios of each operation for different experiments. I
plan to vary read:write:delete and read:update:delete ratios. I am aiming to
have at least eight graphs of different scenarios of these variances. Some I
have planned are (these are planned estimates) for read:write:delete:: (high hit
rate) 1000:10:5, 10:1000:500, (low hit rate) 1000:10:5, 10:1000:500; and for
update:write:delete:: (high hit rate) 1000:10:5, 10:1000:500, (low hit rate)
1000:10:5, 10:1000:500. These are currently just some ideas I have and can
easily come up with different variations as my experiments help me get an idea
of what information may be interesting. The limit on the number of operations /
memory usage is dependant on that which is described in
\ref{subsubsec:max_threshold}.

\subsubsection{Test Hardware}
\label{subsubsec:test_hw}

I will be running these tests on a virtual machine on my home desktop. My home
desktop has a AMD Ryzen R7 1700 CPU, 32 GB 3000MHz DDR4 RAM and a 500 GB primary
storage SSD. The virtual machine will be using (TODO: double check this) 16
CPU cores, 16 GB RAM, and 100 GB of my SSD. Not all 16 GB of RAM will
necessarily be available depending on experiments related to NVRAM which are
discussed in \ref{subsec:opt_rocksdb}.

\subsubsection{Maximum Threshold}
\label{subsubsec:max_threshold}

The range of tests will be as large as possible. Every experiment will take no
longer than 1 minute: any experiment that largely exceeds this amount of time
will not be kept. Note that every experiment will include at least 100 trials
and the 1 minute requirement includes all trials. Tests that complete in
significantly less than 1 minute will accordingly be set to run a larger number
of trials. In addition, obviously any tests that require more than my available
pool of memory and storage will not be run. An idea of what this maximum
threshold may be, will be more clear after reading \ref{subsec:background}.

\subsection{Optimization of RocksDB}
\label{subsec:opt_rocksdb}

I will primarily be trying to optimize RocksDB by making it utilize NVRAM
instead of traditional DRAM. This will be done by using intel's guide to
emulating NVRAM
\footnote{
  \href{https://software.intel.com/en-us/articles/how-to-emulate-persistent-memory-on-an-intel-architecture-server/}
  {Intel NVRAM Emulation Guide}}.
I plan to use 4 GB of RAM as NVRAM. Please refer to \ref{subsubsec:test_hw} to
understand the underlying hardware setup of the system. I will use intel's PMDK
library to utilize the the NVRAM. Specifically, I will be making use of the
libpmemobj\footnote{\href{http://pmem.io/pmdk/libpmemobj/}
{PMDK pmemobj documentation}}.

\section{Level A}

Characterization of major use cases of RocksDB and discussion of optimizations
that would further increase the range of use cases. This involves running
various benchmarks and analysis of the results as decribed in
\ref{subsubsec:characterization_tests}.

\section{Level B}

This involves the work proposed in Level A as well as tests on an optimized
version of RocksDB. This optimization would likely be a modified version of
RocksDB so that it runs on NVRAM as described in \ref{subsec:opt_rocksdb}.

\section{Level C}

This would involve Level A, B, and the application of Levels A and B to LevelDB.
Finally, a comparison of the use cases between the two systems which will simply
include the Level A analysis of both systems. This will also include a
discussion of the cost-benefit of some of the features added to RocksDB which
LevelDB did not have.

\end{document}

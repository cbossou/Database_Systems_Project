\documentclass{article}

\usepackage[utf8]{inputenc}
\usepackage{lipsum}                     % Dummytext
\usepackage{hyperref}
\usepackage{xargs}                      % Use more than one optional parameter in a new commands
\usepackage[pdftex,dvipsnames]{xcolor}  % Coloured text etc.
\usepackage{graphicx}
\usepackage{verbatim}
\usepackage{float}
\usepackage{tikz-qtree}
\usepackage{tikz}
\usepackage[linguistics]{forest}

\usepackage{amssymb}
\usepackage{amsmath}
\newcommand*{\QEDA}{\hfill\ensuremath{\blacksquare}}% filled box
\newcommand*{\QEDB}{\hfill\ensuremath{\square}}% unfilled box

% dem nice tables
\usepackage[hmargin=2cm,top=4cm,headheight=65pt,footskip=65pt]{geometry}
\usepackage{fmtcount} % for \ordinalnum
\usepackage{array,multirow}
\usepackage{tabularx}
\usepackage{lastpage}


% add a special collumn type
\newcolumntype{C}[1]{>{\centering\arraybackslash}m{#1}}


%header/footer stuff
\usepackage{fancyhdr}
\pagestyle{fancy}

%note that if you do not do these blank ones, the package defaults to something
%you may not want in your header or footer
\lhead{Project Proposal}
\chead{CMPS 278}
\rhead{\today}
\lfoot{Isaak Cherdak}
\cfoot{}
\rfoot{\thepage}

\renewcommand{\headrulewidth}{0pt}
\renewcommand{\footrulewidth}{0pt}

\hypersetup{
    colorlinks=true,
    linkcolor=blue,
    filecolor=magenta,
    urlcolor=cyan,
}

\usepackage[english]{babel}
\emergencystretch=1pt
\usepackage[justification=centering]{caption}
\graphicspath{{Pictures/} }

\usepackage[colorinlistoftodos,prependcaption,textsize=tiny]{todonotes}
\newcommandx{\unsure}[2][1=]{\todo[linecolor=red,backgroundcolor=red!25,bordercolor=red,#1]{#2}}
\newcommandx{\change}[2][1=]{\todo[linecolor=blue,backgroundcolor=blue!25,bordercolor=blue,#1]{#2}}
\newcommandx{\info}[2][1=]{\todo[linecolor=OliveGreen,backgroundcolor=OliveGreen!25,bordercolor=OliveGreen,#1]{#2}}
\newcommandx{\improvement}[2][1=]{\todo[linecolor=Plum,backgroundcolor=Plum!25,bordercolor=Plum,#1]{#2}}
\newcommandx{\thiswillnotshow}[2][1=]{\todo[disable,#1]{#2}}

\usepackage{setspace}
\doublespacing

\title{CMPS 278 Project Proposal}
\author{Isaak Cherdak}
%\date{} %blank date

\begin{document}

\maketitle

\pagebreak

\section*{Overview}

I am interested in the prospect of working on characterization of use cases and
potential optimizations of RocksDB. I plan to try to run varying
put/get/update/delete operations on a local RocksDB instance. For optimizations
I am especially interested in converting RocksDB to run on NVRAM. Note that I
have experience emulating and utilizing NVRAM using intel PMDK library. The
size/scope of the operations will depend on what my computer is capable of
handling while also taking a reasonable amount of time per test. I don't believe
I could accurately predict what this would be but in the past I have used up to
40000 for insertions on NVRAM based List, Vector, B-Tree and LSM-Tree data
structures whose data size was one or two 64 bit unsigned integers. In other
words, my upper limit is probably on 40000 inserts of two 64 bit unsigned
integers. The benchmark application itself will simply run these tests for
multiple trials and time each group of operations and write them to respective
files. Multiple tests will be run for different numbers of operations up to some
maximum. An analysis will be performed using pilot, which is a data analysis
tool developed by my lab, the SSRC. These average results (perhaps with error
bars included) will then be plotted using Matplotlib. Statistics recorded will
at minimum include runtimes for various operations as well as various
memory/storage overheads associated with different RocksDB system sizes.

\section*{Level A}

Characterization of major use cases of RocksDB and discussion of optimizations
that would further increase the range of use cases.

\section*{Level B}

Chracterization of major use cases of RocksDB and the additional range of use
cases as a result of implementations of optimizations to RocksDB done as part of
the project.

\section*{Level C}

Comprehensive comparison between RocksDB and one other DB library (i.e LevelDB)
as well as implementation of optimizations to both that further demonstrate the
strengths and weaknesses, particularly of RocksDB. This would also include a
characterization of major use cases of RocksDB as well as the increased range of
use cases as a result of any RocksDB optimizations implemented as part of the
project.

\end{document}

\documentclass{article}

\usepackage[utf8]{inputenc}
\usepackage{lipsum}                     % Dummytext
\usepackage{hyperref}
\usepackage{xargs}                      % Use more than one optional parameter in a new commands
\usepackage[pdftex,dvipsnames]{xcolor}  % Coloured text etc.
\usepackage{graphicx}
\usepackage{verbatim}
\usepackage{float}
\usepackage{tikz-qtree}
\usepackage{tikz}
\usepackage[linguistics]{forest}

\usepackage{amssymb}
\usepackage{amsmath}
\newcommand*{\QEDA}{\hfill\ensuremath{\blacksquare}}% filled box
\newcommand*{\QEDB}{\hfill\ensuremath{\square}}% unfilled box

% dem nice tables
\usepackage[hmargin=2cm,top=4cm,headheight=65pt,footskip=65pt]{geometry}
\usepackage{fmtcount} % for \ordinalnum
\usepackage{array,multirow}
\usepackage{tabularx}
\usepackage{lastpage}


% add a special collumn type
\newcolumntype{C}[1]{>{\centering\arraybackslash}m{#1}}


%header/footer stuff
\usepackage{fancyhdr}
\pagestyle{fancy}

%note that if you do not do these blank ones, the package defaults to something
%you may not want in your header or footer
\lhead{Review for Aurora}
\chead{}
\rhead{\today}
\lfoot{Isaak Cherdak}
\cfoot{}
\rfoot{\thepage}

\renewcommand{\headrulewidth}{0pt}
\renewcommand{\footrulewidth}{0pt}

\hypersetup{
    colorlinks=true,
    linkcolor=blue,
    filecolor=magenta,
    urlcolor=cyan,
}

\usepackage[english]{babel}
\emergencystretch=1pt
\usepackage[justification=centering]{caption}
\graphicspath{{Pictures/} }

\usepackage[colorinlistoftodos,prependcaption,textsize=tiny]{todonotes}
\newcommandx{\unsure}[2][1=]{\todo[linecolor=red,backgroundcolor=red!25,bordercolor=red,#1]{#2}}
\newcommandx{\change}[2][1=]{\todo[linecolor=blue,backgroundcolor=blue!25,bordercolor=blue,#1]{#2}}
\newcommandx{\info}[2][1=]{\todo[linecolor=OliveGreen,backgroundcolor=OliveGreen!25,bordercolor=OliveGreen,#1]{#2}}
\newcommandx{\improvement}[2][1=]{\todo[linecolor=Plum,backgroundcolor=Plum!25,bordercolor=Plum,#1]{#2}}
\newcommandx{\thiswillnotshow}[2][1=]{\todo[disable,#1]{#2}}

\usepackage{setspace}
\doublespacing

\title{Review for Aurora}
\author{Isaak Cherdak}
%\date{} %blank date

\begin{document}

\maketitle

\section{How does the author's approach or solution improve on previous
approaches to the problem that they are solving?}

Previous solutions had focused on smaller scale compute and storage centric
solutions which are often bottlenecked by I/O. The paper discusses how using
aggressive redo logging in a distributed cloud environment can dramatically
increase performance by swapping the bottleneck from I/O to network and
decreasing amplification and stalls which are common in systems that don't
adequately coalesce multiple operations for optimization.

\section{Why is this work important?}

This paper describes a useful way of "eliminating" the bottleneck of I/O and
swapping the concern to network. This is a big deal because up till this point
focus was on improving the performance of individual storage or nodes while this
paper allows us to focus on scaling performance off network instead of storage.

\section{Provide 3+ comments/questions}

\begin{itemize}
  \item Why doesn't Aurora use Write Behind Logging? The WBL paper was published
    the year before this paper.
  \item The paper shows good results for read-only and write-only workloads but
    perhaps these are special cases for many databases. How would Aurora
    perform on various read/write balanced workloads?
  \item Amazon unsurprisingly uses DynamoDB as the database service for its
    implemented Aurora system. How does this compare from the perspective of the
    Aurora database service with using other database services like RocksDB or
    LevelDB?
\end{itemize}

\end{document}

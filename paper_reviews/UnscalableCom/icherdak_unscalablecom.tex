\documentclass{article}

\usepackage[utf8]{inputenc}
\usepackage{lipsum}                     % Dummytext
\usepackage[hyphens]{url}
\usepackage{hyperref}
\usepackage{xargs}                      % Use more than one optional parameter in a new commands
\usepackage[pdftex,dvipsnames]{xcolor}  % Coloured text etc.
\usepackage{graphicx}
\usepackage{verbatim}
\usepackage{float}
\usepackage{tikz-qtree}
\usepackage{tikz}
\usepackage[linguistics]{forest}

\usepackage{amssymb}
\usepackage{amsmath}
\newcommand*{\QEDA}{\hfill\ensuremath{\blacksquare}}% filled box
\newcommand*{\QEDB}{\hfill\ensuremath{\square}}% unfilled box

% dem nice tables
\usepackage[hmargin=2cm,top=4cm,headheight=65pt,footskip=65pt]{geometry}
\usepackage{fmtcount} % for \ordinalnum
\usepackage{array,multirow}
\usepackage{tabularx}
\usepackage{lastpage}


% add a special collumn type
\newcolumntype{C}[1]{>{\centering\arraybackslash}m{#1}}


%header/footer stuff
\usepackage{fancyhdr}
\pagestyle{fancy}

%note that if you do not do these blank ones, the package defaults to something
%you may not want in your header or footer
\lhead{Review for Unscalable Communication}
\chead{}
\rhead{\today}
\lfoot{Isaak Cherdak}
\cfoot{}
\rfoot{\thepage}

\renewcommand{\headrulewidth}{0pt}
\renewcommand{\footrulewidth}{0pt}

\hypersetup{
    colorlinks=true,
    linkcolor=blue,
    filecolor=magenta,
    urlcolor=cyan,
}

\usepackage[english]{babel}
\emergencystretch=1pt
\usepackage[justification=centering]{caption}
\graphicspath{{Pictures/} }

\usepackage[colorinlistoftodos,prependcaption,textsize=tiny]{todonotes}
\newcommandx{\unsure}[2][1=]{\todo[linecolor=red,backgroundcolor=red!25,bordercolor=red,#1]{#2}}
\newcommandx{\change}[2][1=]{\todo[linecolor=blue,backgroundcolor=blue!25,bordercolor=blue,#1]{#2}}
\newcommandx{\info}[2][1=]{\todo[linecolor=OliveGreen,backgroundcolor=OliveGreen!25,bordercolor=OliveGreen,#1]{#2}}
\newcommandx{\improvement}[2][1=]{\todo[linecolor=Plum,backgroundcolor=Plum!25,bordercolor=Plum,#1]{#2}}
\newcommandx{\thiswillnotshow}[2][1=]{\todo[disable,#1]{#2}}

\usepackage{setspace}
\doublespacing

\title{Review for Eliminating Unscalable Communication in transation processing}
\author{Isaak Cherdak}
%\date{} %blank date

\begin{document}

\maketitle

\tableofcontents

\pagebreak

\section{How does the author's approach or solution improve on previous
approaches to the problem that they are solving?}

Some previous work focused on examining the execution of database engine code on
mordern hardware with findings such as out-of-order execution and large caches
being a bad match with most database codes. In comparison this paper also
focused on how to better utilize shared resources which was determined to be of
utmost importance. In general, this paper was a much more high level or abstract
approach to solving issues database systems engineers face compared to those we
had read up till now for this class.

\section{Why is this work important?}

This paper helps identify three forms of communication in a system: unbounded,
fixed, and cooperative. These identifications lead to database designers being
able to make better choices about system design and SLA by virtue of having a
better conceptual understanding of how a system scales depending on the high
level rules imposed.

\section{Provide 3+ comments/questions}

\begin{itemize}
  \item I really liked this paper. It was very different from the previous
    papers in this class (a breathe of fresh air). Though what I also really
    liked is how it discusses many different advanced topics that various
    database systems use and their implications: almost like a primer to
    various advanced database design techniques.
  \item This paper makes a broader point about different database systems
    techniques and how they affect different database systems in different
    benchmarks (workloads).
  \item Another things this paper did that other papers should do more often is
    really discuss workloads. Sure, it was a large focus of the paper but as was
    discussed in class, a database system should be designed for a specific
    workload. Unfortunately, many papers don't spend enough time clarifying
    the nuances of their workload and why their database system is a
    good fit.
\end{itemize}

\end{document}

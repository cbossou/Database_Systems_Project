\documentclass{article}

\usepackage[utf8]{inputenc}
\usepackage{lipsum}                     % Dummytext
\usepackage[hyphens]{url}
\usepackage{hyperref}
\usepackage{xargs}                      % Use more than one optional parameter in a new commands
\usepackage[pdftex,dvipsnames]{xcolor}  % Coloured text etc.
\usepackage{graphicx}
\usepackage{verbatim}
\usepackage{float}
\usepackage{tikz-qtree}
\usepackage{tikz}
\usepackage[linguistics]{forest}

\usepackage{amssymb}
\usepackage{amsmath}
\newcommand*{\QEDA}{\hfill\ensuremath{\blacksquare}}% filled box
\newcommand*{\QEDB}{\hfill\ensuremath{\square}}% unfilled box

% dem nice tables
\usepackage[hmargin=2cm,top=4cm,headheight=65pt,footskip=65pt]{geometry}
\usepackage{fmtcount} % for \ordinalnum
\usepackage{array,multirow}
\usepackage{tabularx}
\usepackage{lastpage}


% add a special collumn type
\newcolumntype{C}[1]{>{\centering\arraybackslash}m{#1}}


%header/footer stuff
\usepackage{fancyhdr}
\pagestyle{fancy}

%note that if you do not do these blank ones, the package defaults to something
%you may not want in your header or footer
\lhead{Review for DCC}
\chead{}
\rhead{\today}
\lfoot{Isaak Cherdak}
\cfoot{}
\rfoot{\thepage}

\renewcommand{\headrulewidth}{0pt}
\renewcommand{\footrulewidth}{0pt}

\hypersetup{
    colorlinks=true,
    linkcolor=blue,
    filecolor=magenta,
    urlcolor=cyan,
}

\usepackage[english]{babel}
\emergencystretch=1pt
\usepackage[justification=centering]{caption}
\graphicspath{{Pictures/} }

\usepackage[colorinlistoftodos,prependcaption,textsize=tiny]{todonotes}
\newcommandx{\unsure}[2][1=]{\todo[linecolor=red,backgroundcolor=red!25,bordercolor=red,#1]{#2}}
\newcommandx{\change}[2][1=]{\todo[linecolor=blue,backgroundcolor=blue!25,bordercolor=blue,#1]{#2}}
\newcommandx{\info}[2][1=]{\todo[linecolor=OliveGreen,backgroundcolor=OliveGreen!25,bordercolor=OliveGreen,#1]{#2}}
\newcommandx{\improvement}[2][1=]{\todo[linecolor=Plum,backgroundcolor=Plum!25,bordercolor=Plum,#1]{#2}}
\newcommandx{\thiswillnotshow}[2][1=]{\todo[disable,#1]{#2}}

\usepackage{setspace}
\doublespacing

\title{Review for DCC}
\author{Isaak Cherdak}
%\date{} %blank date

\begin{document}

\maketitle

\tableofcontents

\pagebreak

\section{How does the author's approach or solution improve on previous
approaches to the problem that they are solving?}

Unlike previous work, this paper compares many different data base management
systems and hence many different distributed concurrency control systems. Toward
this goal, the authors made a system called Deneva used to evaluate various DBMS
 catered toward comparing the concurrency control system in each DBMS.

\section{Why is this work important?}

This paper gives an analysis of the trade-offs of distributed concurency control
in modern cloud computing to offer the best solutions for scalability.

\section{Provide 3+ comments/questions}

\begin{itemize}
  \item It would be nice if there was some sort of way to determine "how
    distributed" a database management system should be, based off its use case,
    to be optimal.
  \item Partitioning the database is a clever way to isolate investigation of
    scalability bottlenecks to the concurrency control system since more
    partitions mean that more time must be spent performing concurrency control
    to ensure consistency between them.
  \item The graphs are really nice in this paper. Also, what's with "other" on
    figure 8? It takes up most of the results.
\end{itemize}

\end{document}

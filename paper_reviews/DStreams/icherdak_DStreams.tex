\documentclass{article}

\usepackage[utf8]{inputenc}
\usepackage{lipsum}                     % Dummytext
\usepackage[hyphens]{url}
\usepackage{hyperref}
\usepackage{xargs}                      % Use more than one optional parameter in a new commands
\usepackage[pdftex,dvipsnames]{xcolor}  % Coloured text etc.
\usepackage{graphicx}
\usepackage{verbatim}
\usepackage{float}
\usepackage{tikz-qtree}
\usepackage{tikz}
\usepackage[linguistics]{forest}

\usepackage{amssymb}
\usepackage{amsmath}
\newcommand*{\QEDA}{\hfill\ensuremath{\blacksquare}}% filled box
\newcommand*{\QEDB}{\hfill\ensuremath{\square}}% unfilled box

% dem nice tables
\usepackage[hmargin=2cm,top=4cm,headheight=65pt,footskip=65pt]{geometry}
\usepackage{fmtcount} % for \ordinalnum
\usepackage{array,multirow}
\usepackage{tabularx}
\usepackage{lastpage}


% add a special collumn type
\newcolumntype{C}[1]{>{\centering\arraybackslash}m{#1}}


%header/footer stuff
\usepackage{fancyhdr}
\pagestyle{fancy}

%note that if you do not do these blank ones, the package defaults to something
%you may not want in your header or footer
\lhead{Review for DStreams}
\chead{}
\rhead{\today}
\lfoot{Isaak Cherdak}
\cfoot{}
\rfoot{\thepage}

\renewcommand{\headrulewidth}{0pt}
\renewcommand{\footrulewidth}{0pt}

\hypersetup{
    colorlinks=true,
    linkcolor=blue,
    filecolor=magenta,
    urlcolor=cyan,
}

\usepackage[english]{babel}
\emergencystretch=1pt
\usepackage[justification=centering]{caption}
\graphicspath{{Pictures/} }

\usepackage[colorinlistoftodos,prependcaption,textsize=tiny]{todonotes}
\newcommandx{\unsure}[2][1=]{\todo[linecolor=red,backgroundcolor=red!25,bordercolor=red,#1]{#2}}
\newcommandx{\change}[2][1=]{\todo[linecolor=blue,backgroundcolor=blue!25,bordercolor=blue,#1]{#2}}
\newcommandx{\info}[2][1=]{\todo[linecolor=OliveGreen,backgroundcolor=OliveGreen!25,bordercolor=OliveGreen,#1]{#2}}
\newcommandx{\improvement}[2][1=]{\todo[linecolor=Plum,backgroundcolor=Plum!25,bordercolor=Plum,#1]{#2}}
\newcommandx{\thiswillnotshow}[2][1=]{\todo[disable,#1]{#2}}

\usepackage{setspace}
\doublespacing

\title{Review for DStreams}
\author{Isaak Cherdak}
%\date{} %blank date

\begin{document}

\maketitle

\section{How does the author's approach or solution improve on previous
approaches to the problem that they are solving?}

D-streams present the time that each task will take in advance which leads to
very powerful semantics regarding recovery and the handling of stragglers.

\section{Why is this work important?}

It shows that it may be well worth increasing minimum latency if you enable more
efficient handling of special scenarios that normally cost traditional systems a
lot of resources to handle if at all. Namely this paper is the first that I have
seen where stragglers are attempted to be systematically handled at all.

\section{Provide 3+ comments/questions}

\begin{itemize}
  \item The paper seems to take a lot from the work that Spark did. In doing so,
    it also makes it difficult to see what is a novel idea in D-streams and what
    is simply a property in RDDs.
  \item Hadoop is mentioned as keeping state on replicated storage between jobs
    which is described as a traditional batch system short fall. The paper
    however goes on to say that data can be placed on storage using HDFS which
    seems to be contradictory given what I mentioned in the previous sentence.
    Isn't there a more appropriate file system that DStreams could be based off
    of?
  \item Figure 9 is concerning to me. It looks like with larger record sizes,
    storm's performance improves but Spark Streaming doesn't improve at all. I
    see that these record sizes allow Spark Streaming to outperform Storm but
    based off this trend I would like to see the result if the record size
    continued to increase.
\end{itemize}

\end{document}

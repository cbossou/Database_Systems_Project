\documentclass{article}

\usepackage[utf8]{inputenc}
\usepackage{lipsum}                     % Dummytext
\usepackage[hyphens]{url}
\usepackage{hyperref}
\usepackage{xargs}                      % Use more than one optional parameter in a new commands
\usepackage[pdftex,dvipsnames]{xcolor}  % Coloured text etc.
\usepackage{graphicx}
\usepackage{verbatim}
\usepackage{float}
\usepackage{tikz-qtree}
\usepackage{tikz}
\usepackage[linguistics]{forest}

\usepackage{amssymb}
\usepackage{amsmath}
\newcommand*{\QEDA}{\hfill\ensuremath{\blacksquare}}% filled box
\newcommand*{\QEDB}{\hfill\ensuremath{\square}}% unfilled box

% dem nice tables
\usepackage[hmargin=2cm,top=4cm,headheight=65pt,footskip=65pt]{geometry}
\usepackage{fmtcount} % for \ordinalnum
\usepackage{array,multirow}
\usepackage{tabularx}
\usepackage{lastpage}


% add a special collumn type
\newcolumntype{C}[1]{>{\centering\arraybackslash}m{#1}}


%header/footer stuff
\usepackage{fancyhdr}
\pagestyle{fancy}

%note that if you do not do these blank ones, the package defaults to something
%you may not want in your header or footer
\lhead{Review for MapReduce}
\chead{}
\rhead{\today}
\lfoot{Isaak Cherdak}
\cfoot{}
\rfoot{\thepage}

\renewcommand{\headrulewidth}{0pt}
\renewcommand{\footrulewidth}{0pt}

\hypersetup{
    colorlinks=true,
    linkcolor=blue,
    filecolor=magenta,
    urlcolor=cyan,
}

\usepackage[english]{babel}
\emergencystretch=1pt
\usepackage[justification=centering]{caption}
\graphicspath{{Pictures/} }

\usepackage[colorinlistoftodos,prependcaption,textsize=tiny]{todonotes}
\newcommandx{\unsure}[2][1=]{\todo[linecolor=red,backgroundcolor=red!25,bordercolor=red,#1]{#2}}
\newcommandx{\change}[2][1=]{\todo[linecolor=blue,backgroundcolor=blue!25,bordercolor=blue,#1]{#2}}
\newcommandx{\info}[2][1=]{\todo[linecolor=OliveGreen,backgroundcolor=OliveGreen!25,bordercolor=OliveGreen,#1]{#2}}
\newcommandx{\improvement}[2][1=]{\todo[linecolor=Plum,backgroundcolor=Plum!25,bordercolor=Plum,#1]{#2}}
\newcommandx{\thiswillnotshow}[2][1=]{\todo[disable,#1]{#2}}

\usepackage{setspace}
\doublespacing

\title{Review for MapReduce}
\author{Isaak Cherdak}
%\date{} %blank date

\begin{document}

\maketitle

\section{How does the author's approach or solution improve on previous
approaches to the problem that they are solving?}

Google realizes that many kinds of problems can all be structured as
``MapReduce". MapReduce as an approach is very simple: to such a degree
that google decided to try parallelizing it on lots of nodes. This kind of
approach to handling large datasets was not yet largely used as of publish time.

\section{Why is this work important?}

MapReduce demonstrates how useful it can be to be able to generalize a large
portion of computations/algorithms. It is the first of its kind in completeness
and scale: as the paper mentions, no implementation was yet able to handle this
level of workload or fault tolerance as of publish time.

\section{Provide 3+ comments/questions}

\begin{itemize}
  \item It seems that google is moving to Cloud Dataflow since MapReduce ``does
    not handle petabyte-scale analytics well enough"
    (\url{http://www.datacenterknowledge.com/archives/2014/06/25/google-dumps-mapreduce-favor-new-hyper-scale-analytics-system}).
    I already could tell that they had a completely different situation than the
    one industry would face today when I read that the bottleneck for MapReduce
    was the network while storage was based off IDE hard drives. Sure enough, a
    google search was enough to tell me the approach was not scalable enough for
    modern tech.
  \item From my understanding the programming model was publicly available but
    the parallel processing center implementation of MapReduce that google had
    was not. Why didn't google decide to offer external use of MapReduce as a
    paid serivce?
  \item To me, this is another example of a paper that doesn't really introduce
    anything very new but twists a few small details in ways that others had not
    yet done. That being said, this is very much like the RocksDB paper in that
    it actually discusses a real world system. Not only does the paper provide
    useful insights into applications that are most common for MapReduce but
    also useful lessons about scalability problems like shifting bottleneck away
    from network bandwidth and handling straggling computational units.
\end{itemize}

\end{document}

\documentclass{article}

\usepackage[utf8]{inputenc}
\usepackage{lipsum}                     % Dummytext
\usepackage{hyperref}
\usepackage{xargs}                      % Use more than one optional parameter in a new commands
\usepackage[pdftex,dvipsnames]{xcolor}  % Coloured text etc.
\usepackage{graphicx}
\usepackage{verbatim}
\usepackage{float}
\usepackage{tikz-qtree}
\usepackage{tikz}
\usepackage[linguistics]{forest}

\usepackage{amssymb}
\usepackage{amsmath}
\newcommand*{\QEDA}{\hfill\ensuremath{\blacksquare}}% filled box
\newcommand*{\QEDB}{\hfill\ensuremath{\square}}% unfilled box

% dem nice tables
\usepackage[hmargin=2cm,top=4cm,headheight=65pt,footskip=65pt]{geometry}
\usepackage{fmtcount} % for \ordinalnum
\usepackage{array,multirow}
\usepackage{tabularx}
\usepackage{lastpage}


% add a special collumn type
\newcolumntype{C}[1]{>{\centering\arraybackslash}m{#1}}


%header/footer stuff
\usepackage{fancyhdr}
\pagestyle{fancy}

%note that if you do not do these blank ones, the package defaults to something
%you may not want in your header or footer
\lhead{Review for LSM}
\chead{}
\rhead{\today}
\lfoot{Isaak Cherdak}
\cfoot{}
\rfoot{\thepage}

\renewcommand{\headrulewidth}{0pt}
\renewcommand{\footrulewidth}{0pt}

\hypersetup{
    colorlinks=true,
    linkcolor=blue,
    filecolor=magenta,
    urlcolor=cyan,
}

\usepackage[english]{babel}
\emergencystretch=1pt
\usepackage[justification=centering]{caption}
\graphicspath{{Pictures/} }

\usepackage[colorinlistoftodos,prependcaption,textsize=tiny]{todonotes}
\newcommandx{\unsure}[2][1=]{\todo[linecolor=red,backgroundcolor=red!25,bordercolor=red,#1]{#2}}
\newcommandx{\change}[2][1=]{\todo[linecolor=blue,backgroundcolor=blue!25,bordercolor=blue,#1]{#2}}
\newcommandx{\info}[2][1=]{\todo[linecolor=OliveGreen,backgroundcolor=OliveGreen!25,bordercolor=OliveGreen,#1]{#2}}
\newcommandx{\improvement}[2][1=]{\todo[linecolor=Plum,backgroundcolor=Plum!25,bordercolor=Plum,#1]{#2}}
\newcommandx{\thiswillnotshow}[2][1=]{\todo[disable,#1]{#2}}

\usepackage{setspace}
\doublespacing

\title{Review for LSM}
\author{Isaak Cherdak}
%\date{} %blank date

\begin{document}

\maketitle

\section{How does the author's approach or solution improve on previous
approaches to the problem that they are solving?}

Previous data structures had large amounts of write amplification if they were
to manage data at scale. This was especially a problem since NVRAM was not yet
available and flushes to disk would occur often hence causing serious wear on
storage. An LSM tree approaches this by initially buffering writes in a memory
tree component, thus making batch writes faster and have less wear on storage.

\section{Why is this work important?}

This was initially very helpful when disk storage was prevalent and enabled
significant decreases in wear due to the on-memory buffer. This also set the
ground work for database systems like Facebook RocksDB and other highly
write-optimized systems.

\section{Provide 3+ comments/questions}

\begin{itemize}
  \item Why did it take such a long time for the LSM tree to gain traction and
    have applications like Cassandra and RocksDB.
  \item What additional optimizations can be made if a LSM tree is optimized
    with NVRAM? My project may give insight to this.
  \item Why aren't more popular systems using this datastructure now given that
    RocksDB which really helped with its popularity came out in 2012.
\end{itemize}

\end{document}

\documentclass{article}

\usepackage[utf8]{inputenc}
\usepackage{lipsum}                     % Dummytext
\usepackage{hyperref}
\usepackage{xargs}                      % Use more than one optional parameter in a new commands
\usepackage[pdftex,dvipsnames]{xcolor}  % Coloured text etc.
\usepackage{graphicx}
\usepackage{verbatim}
\usepackage{float}
\usepackage{tikz-qtree}
\usepackage{tikz}
\usepackage[linguistics]{forest}

\usepackage{amssymb}
\usepackage{amsmath}
\newcommand*{\QEDA}{\hfill\ensuremath{\blacksquare}}% filled box
\newcommand*{\QEDB}{\hfill\ensuremath{\square}}% unfilled box

% dem nice tables
\usepackage[hmargin=2cm,top=4cm,headheight=65pt,footskip=65pt]{geometry}
\usepackage{fmtcount} % for \ordinalnum
\usepackage{array,multirow}
\usepackage{tabularx}
\usepackage{lastpage}


% add a special collumn type
\newcolumntype{C}[1]{>{\centering\arraybackslash}m{#1}}


%header/footer stuff
\usepackage{fancyhdr}
\pagestyle{fancy}

%note that if you do not do these blank ones, the package defaults to something
%you may not want in your header or footer
\lhead{Review for Write Behind Logging}
\chead{}
\rhead{\today}
\lfoot{Isaak Cherdak}
\cfoot{}
\rfoot{\thepage}

\renewcommand{\headrulewidth}{0pt}
\renewcommand{\footrulewidth}{0pt}

\hypersetup{
    colorlinks=true,
    linkcolor=blue,
    filecolor=magenta,
    urlcolor=cyan,
}

\usepackage[english]{babel}
\emergencystretch=1pt
\usepackage[justification=centering]{caption}
\graphicspath{{Pictures/} }

\usepackage[colorinlistoftodos,prependcaption,textsize=tiny]{todonotes}
\newcommandx{\unsure}[2][1=]{\todo[linecolor=red,backgroundcolor=red!25,bordercolor=red,#1]{#2}}
\newcommandx{\change}[2][1=]{\todo[linecolor=blue,backgroundcolor=blue!25,bordercolor=blue,#1]{#2}}
\newcommandx{\info}[2][1=]{\todo[linecolor=OliveGreen,backgroundcolor=OliveGreen!25,bordercolor=OliveGreen,#1]{#2}}
\newcommandx{\improvement}[2][1=]{\todo[linecolor=Plum,backgroundcolor=Plum!25,bordercolor=Plum,#1]{#2}}
\newcommandx{\thiswillnotshow}[2][1=]{\todo[disable,#1]{#2}}

\usepackage{setspace}
\doublespacing

\title{Review for Write Behind Logging (WBL)}
\author{Isaak Cherdak}
%\date{} %blank date

\begin{document}

\maketitle

\section{How does the author's approach or solution improve on previous
approaches to the problem that they are solving?}

The author determines that Write Ahead Logging (WAL) is no longer the best
method for Data Base Management System (DBMS) consistency with the advent of
byte-addressable NVM (BNVM). Specifically, this previous approach was based off
the technologies available prior to BNVM. In response, the author introduces
WBL. In other words, this is more of a modern replacement than an improvement.

\section{Why is this work important?}

This paper is a milestone in technological change. It marks a time where DBMSs
will have access to a new technology: byte-addressable NVM. Further, it
discusses some of the important new considerations that must be made when
effectively making use of NVM as well as describing a new consistency method to
replace WAL: WBL.

\section{Provide 3+ comments/questions}

\begin{itemize}
  \item Are there any systems making use of WBL commercially today? I am sure
    some companies have the money to afford BNVM today.
  \item The evaluation section of this paper mentioned some specific NVM latency
    timings which I believe may have introduced inaccuracies into the results.
    Firstly, BNVM latencies are improving. In addition, intel has a pmem
    utilization library called PMDK (formerly called NVML) that handles these
    latencies for you and has it's own set of persistent memory allocation
    mechanics and transactions that it provides. I wonder how significant the
    change to the results may be if these two things are integrated/reconsidered
    in this paper.
  \item The results don't seem to reflect what I expected. It seems that the
    performance of WAL is comparable to WBL in general performance. Only
    recovery time is significantly better in WBL accross the board.
\end{itemize}

\end{document}

\documentclass{article}

\usepackage[utf8]{inputenc}
\usepackage{lipsum}                     % Dummytext
\usepackage[hyphens]{url}
\usepackage{hyperref}
\usepackage{xargs}                      % Use more than one optional parameter in a new commands
\usepackage[pdftex,dvipsnames]{xcolor}  % Coloured text etc.
\usepackage{graphicx}
\usepackage{verbatim}
\usepackage{float}
\usepackage{tikz-qtree}
\usepackage{tikz}
\usepackage[linguistics]{forest}

\usepackage{amssymb}
\usepackage{amsmath}
\newcommand*{\QEDA}{\hfill\ensuremath{\blacksquare}}% filled box
\newcommand*{\QEDB}{\hfill\ensuremath{\square}}% unfilled box

% dem nice tables
\usepackage[hmargin=2cm,top=4cm,headheight=65pt,footskip=65pt]{geometry}
\usepackage{fmtcount} % for \ordinalnum
\usepackage{array,multirow}
\usepackage{tabularx}
\usepackage{lastpage}


% add a special collumn type
\newcolumntype{C}[1]{>{\centering\arraybackslash}m{#1}}


%header/footer stuff
\usepackage{fancyhdr}
\pagestyle{fancy}

%note that if you do not do these blank ones, the package defaults to something
%you may not want in your header or footer
\lhead{Review for RDD}
\chead{}
\rhead{\today}
\lfoot{Isaak Cherdak}
\cfoot{}
\rfoot{\thepage}

\renewcommand{\headrulewidth}{0pt}
\renewcommand{\footrulewidth}{0pt}

\hypersetup{
    colorlinks=true,
    linkcolor=blue,
    filecolor=magenta,
    urlcolor=cyan,
}

\usepackage[english]{babel}
\emergencystretch=1pt
\usepackage[justification=centering]{caption}
\graphicspath{{Pictures/} }

\usepackage[colorinlistoftodos,prependcaption,textsize=tiny]{todonotes}
\newcommandx{\unsure}[2][1=]{\todo[linecolor=red,backgroundcolor=red!25,bordercolor=red,#1]{#2}}
\newcommandx{\change}[2][1=]{\todo[linecolor=blue,backgroundcolor=blue!25,bordercolor=blue,#1]{#2}}
\newcommandx{\info}[2][1=]{\todo[linecolor=OliveGreen,backgroundcolor=OliveGreen!25,bordercolor=OliveGreen,#1]{#2}}
\newcommandx{\improvement}[2][1=]{\todo[linecolor=Plum,backgroundcolor=Plum!25,bordercolor=Plum,#1]{#2}}
\newcommandx{\thiswillnotshow}[2][1=]{\todo[disable,#1]{#2}}

\usepackage{setspace}
\doublespacing

\title{Review for Resilient Distributed Datasets (RDDs)}
\author{Isaak Cherdak}
%\date{} %blank date

\begin{document}

\maketitle

\section{How does the author's approach or solution improve on previous
approaches to the problem that they are solving?}

It was largely unpopular to try to perform large scale computations mostly in
memory due to concerns of fault tolerance. This paper discusses an abstraction,
RDD, which attempts to accomplish fault tolerant large scale computations in
memory. RDD also uses coarse-grained transformations of data rather than the
previously common fine-grained updates.

\section{Why is this work important?}

This paper discusses not only an abstraction for performing fault tolerant large
scale computation in memory but also an implementation, Spark, in which the
abstraction is demonstrated to be applicable and functional as claimed.

\section{Provide 3+ comments/questions}

\begin{itemize}
  \item This paper has given me a hint of a possible pattern for success. The
    model seems to be to try to design a system that is able to do things entire
    previous systems promise, but even more. MapReduce demonstrated how many
    different problems can be simplified to MapReduce problems, and yet this
    paper discusses how one application, of the infinitely many applications, is
    to use RDD as MapReduce. I still question the truth of claims like this as
    the fine-grained differences will still exist when trying to generalize
    applications like this but I do see how from a progressive perspective it is
    overall easier and more useful.
  \item I find the concept of using lineage to be an interesting and novel
    approach to an abstraction rather than code implementation: where the term
    is inheritance.
  \item Just as the trend has been going, I believe that someone will come along
    and demonstrate how coarse-grained transformations can be used for things
    like storage systems contrary to what this paper claims. Such a paper will
    then be "the next big thing".
\end{itemize}

\end{document}
